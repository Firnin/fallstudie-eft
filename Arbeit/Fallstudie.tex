\documentclass[12pt, a4paper]{article}
\usepackage[german]{babel}

\title{Exposé - Einführung und Digitalisierung eines neuen Geschäftsprozesses}
\author{Kilian Schlosser}
\date{Mai 2023}

\begin{document}


\maketitle

\begin{abstract}
    tbd
\end{abstract}

\tableofcontents

\section{Einleitung}

Diese Arbeit beschäftigt sich mit der Einführung eines neuen Geschäftsprozesses zur Inventarisierung und Sichtprüfung von Electronic Funds Transfers-Terminals (EFTs). 
Das Ziel besteht darin, den bestehenden manuellen Prozess zu optimieren und durch eine digitale Lösung zu ersetzen. \textit{Diese Einleitung gibt einen Überblick über die 
Aufgabenstellung, die Einbettung des Themas in ein größeres Umfeld und bietet einen Ausblick auf die weiteren Kapitel dieser Arbeit.}
Die Arbeit findet im Rahmen eines größeren Projekts innerhalb der tegut... gute Lebensmittel GmbH \& Co. KG (nachfolgend nur tegut genannt) statt. Das Projekt hat die Einführung
einer Ende-zu-Ende-Verschlüsselung von Kreditkartenkarten vom Bezahlterminal zum Zahlungsdienstleister als Ziel. Damit die Firmware, die das ermöglicht, auf den Bezahlterminals
eingesetzt werden darf, muss tegut eine neue Zertifizierungsstufe (genannt \textbf{nachschlagen und einfügen}) ablegen. Um diese erreichen zu können, müssen einige Anforderungen
in einem Selbstüberprüfungsprozess gegeben sein, unter Anderem eine regelmäßige Inventarisierung und Sichtprüfung aller aktiv eingesetzten EFTs.

Die Arbeit ist in mehrere Kapitel unterteilt, in denen verschiedene Aspekte der Problemlösung näher beschrieben werden.

\end{document}