\documentclass[12pt, a4paper]{article}
\usepackage[german]{babel}

\title{Einführung und Digitalisierung eines neuen Geschäftsprozesses - Sichtprüfung und Inventur von EFTs}
\author{Kilian Schlosser}
\date{Sommersemester 2023}

\begin{document}


\maketitle


\tableofcontents
\newpage

\section{Einleitung}

Diese Arbeit beschäftigt sich mit der Einführung eines neuen Geschäftsprozesses zur Inventarisierung und Sichtprüfung von Electronic Funds Transfers (EFTs). 
Das Ziel besteht darin, den bestehenden manuellen Prozess zu optimieren und durch eine digitale Lösung zu ersetzen. Diese Einleitung gibt einen Überblick über die 
Aufgabenstellung, die Einbettung des Themas in ein größeres Umfeld und bietet einen Ausblick auf die weiteren Kapitel dieser Arbeit.

Die zunehmende Digitalisierung und Automatisierung von Geschäftsprozessen hat in den letzten Jahren zu erheblichen Verbesserungen in Effizienz und Genauigkeit geführt. \textbf{Nachweis suchen!}
Die Inventarisierung und Sichtprüfung von EFTs stellt jedoch weiterhin eine Herausforderung dar, da der aktuelle Prozess manuell und papierbasiert ist und nicht regelmäßig durchgeführt wird. 
Um diese Schwachstellen zu beheben, wird in dieser Arbeit ein neuer Geschäftsprozess eingeführt, der die Vorteile der Digitalisierung nutzt.
Die Motivation dahinter stammt aus einem Projekt der tegut... gute Lebensmittel GmbH \& Co. KG (nachfolgend nur kurz tegut... genannt) zur Verschlüsselung der Übertragung von
Kreditkartendaten vom Bezahlterminal zum Zahlungsdienstleister. Um die Firmware einsetzen zu können, die diese Verschlüsselung ermöglicht, muss tegut... eine höhere Stufe der 
"Payment Card Industry Data Security Standard"-Zertifizierung (PCI-DSS) als bislang erreichen. Um dieser gerecht zu werden, muss unter Anderem eine laufende Inventur und
Sichtprüfung aller aktiv eingesetzten EFTs gewährleistet sein. 

Die Arbeit ist in mehrere Kapitel unterteilt, die jeweils verschiedene Aspekte der Problematik behandeln. Im zweiten Kapitel wird die Planungsphase näher beschrieben, 
in der der IST-Prozess analysiert und Schwachstellen identifiziert werden. Dabei wird auch die Wahl der Bizagi-Plattform als Lösung zur Digitalisierung des Prozesses erläutert. 
Das dritte Kapitel widmet sich der Modellierung des SOLL-Prozesses und zeigt auf, wie die Inventarisierung und Sichtprüfung der EFTs digitalisiert und automatisiert werden 
können.

Im vierten Kapitel wird die Einführung und Abstimmung des neuen Geschäftsprozesses diskutiert. Es werden die verschiedenen Stakeholder einbezogen und deren Bedürfnisse 
berücksichtigt, um eine reibungslose Implementierung zu gewährleisten. Des Weiteren wird im fünften Kapitel die Frage der Endgeräte diskutiert und die Vor- und Nachteile 
von Mobilen Datenerfassungsgeräten (MDE) und Personal Computern (PC) abgewogen.

Abschließend werden im sechsten Kapitel die wichtigsten Erkenntnisse zusammengefasst und ein Ausblick auf mögliche zukünftige Entwicklungen gegeben.

\section{Planungsphase}

Eine entscheidende Phase vor der eigentlichen Implementierung des neuen Geschäftsprozesses war die Planungsphase. Hier wurden grundlegende Überlegungen und Entscheidungen 
getroffen, die die Basis für die spätere Umsetzung bildeten. Ein Hauptaugenmerk lag dabei auf der Wahl des optimalen Endgeräts für den Prozess.
Zusätzlich besteht sie aus einer detaillierten Analyse des IST-Prozesses, Identifikation der Schwachstellen und einer gezielten Lösungssuche.

\subsection{Analyse des IST-Prozesses}

Eine detaillierte Aufnahme des bestehenden Prozesses war essentiell, um die aktuellen Abläufe, Verantwortlichkeiten und Tools zu verstehen. Hierbei wurden folgende Punkte 
festgestellt:
\begin{itemize}
\item Die EFTs wurden bei Lieferung manuell in Listen erfasst und mit Inventaraufklebern versehen.
\item Anschließend wurde die Liste in die Content Management Database (CMDB) importiert.
\item Bei Auslieferung an eine Filiale wurde die Filial- und Kassennummer nachgetragen.
\item Die Abläufe waren nicht fest definiert.
\end{itemize}
Einen Prozess zur laufenden Inventur oder einer Sichtprüfung gab es bislang nicht.

\subsection{Identifikation von Schwachstellen}

Die Analyse hat mehrere Schwachstellen des IST-Prozesses offengelegt:
\begin{itemize}
\item Zeitaufwand: Der manuelle Prozess ist zeitaufwändig und erfordert mehrere Mitarbeiter.
\item Nachverfolgbarkeit: Ohne eine regelmäßige Inventur ist es schwierig, einen Überblick über alle EFTs zu behalten und sie nachzuverfolgen.
\item Reproduzierbarkeit: Ohne festgeschriebene Abläufe kann der Prozess nicht jedes Mal korrekt reproduziert werden.
\end{itemize}
Der Prozess zur Aufnahme der EFTs in das Inventar, ist ein Standardprozess für alle neue Hardware bei tegut... . Diesen Prozess neu zu gestalten ist nicht Bestandteil dieser
Arbeit und wird bei tegut... bereits in einem separatem Projekt bearbeitet.

\subsection{Wahl der Bizagi-Plattform}

Nach einer gründlichen Marktanalyse wurde die Bizagi-Plattform schon vor einigen Jahren von tegut... als ideale Lösung zur Digitalisierung von Geschäftsprozessen gewählt. 
Einige Vorteile dieser Plattform sind:
\begin{itemize}
\item Benutzerfreundlichkeit: Die Plattform bietet eine intuitive Benutzeroberfläche, die den Übergang von einem manuellen zu einem digitalen Prozess erleichtert.
\item Flexibilität: Die Plattform ermöglicht es, den Prozess nach Bedarf anzupassen und zu erweitern.
\end{itemize}
Die Implementierung des Prozesses mit Hilfe der Bizagi-Plattform wird im nächsten Kapitel näher erläutert.

\subsection{Mobil vs. Stationär: Die Wahl des Endgeräts}

Die Verwendung eines geeigneten Endgeräts ist entscheidend für den Erfolg des digitalisierten Prozesses. Insbesondere ging es darum, ob moderne MDEs oder herkömmliche PCs eingesetzt werden sollten.

\subsubsection{Mobile Datenerfassungsgeräte (MDEs)}

\textbf{Vorteile:}
\begin{itemize}
\item \textit{Mobilität:} MDEs können überall hin mitgenommen werden, was gerade bei der Erfassung von EFTs vor Ort sehr nützlich ist.
\item \textit{Echtzeit-Erfassung:} Daten können sofort vor Ort eingegeben und übertragen werden.
\item \textit{Barcodescanner und Kamera:} Die MDEs sind mit Barcodescannern und Kameras ausgestattet, was die Datenerfassung erleichtert.
\end{itemize}

\textbf{Nachteile:}
\begin{itemize}
\item \textit{Begrenzte Bildschirmgröße:} Die Erfassung komplexer Daten kann schwieriger sein als auf einem PC.
\end{itemize}

\subsubsection{Personal Computer (PC)}

\textbf{Vorteile:}
\begin{itemize}
\item \textit{Größerer Bildschirm:} Dies ermöglicht eine übersichtlichere Darstellung und leichtere Eingabe komplexer Daten.
\item \textit{Leistungsfähigkeit:} PCs können komplexe Aufgaben schneller verarbeiten.
\item \textit{Standardisierung:} PCs sind bereits in den Büros der Filialen vorhanden und die Mitarbeiter sind mit ihrer Bedienung vertraut.
\end{itemize}

\textbf{Nachteile:}
\begin{itemize}
\item \textit{Mobilität:} PCs sind in der Regel stationär und nicht für den Einsatz vor Ort geeignet. Das erhöht die Fehlerquote bei der Übertragung von komplexen Daten,
wie Inventar- oder Seriennummern.
\end{itemize}

\subsubsection{Entscheidung}
Nach sorgfältiger Abwägung aller Vor- und Nachteile beider Lösungen wurde in Absprache mit dem Enterprise Architekten der tegut...-Informationstechnologie entschieden, 
den Prozess zunächst für den PC zu modellieren. Diese Entscheidung wurde maßgeblich von der Tatsache beeinflusst, dass noch nicht alle Filialen die neuen, Android-betriebenen 
MDEs verwenden. Zudem legt der Enterprise Architekt Wert darauf, dass die MDEs als Plattform so isoliert wie möglich betrieben werden.

\section{Modellierung des SOLL-Prozesses}

Nachdem die Schwachstellen des IST-Prozesses identifiziert und die Bizagi-Plattform als Lösung ausgewählt wurde, folgt nun die Modellierung des gewünschten (SOLL) Prozesses. 
Dieser Abschnitt beschreibt die Schritte und Überlegungen, die in diese Phase eingeflossen sind.

\subsection{Ziele des SOLL-Prozesses}

Zunächst wurden die Hauptziele für den neuen Prozess definiert:
\begin{itemize}
\item Minimierung des manuellen Aufwands durch Automatisierung.
\item Erhöhung der Transparenz und Nachverfolgbarkeit von EFTs.
\item Sicherstellung der Datenkonsistenz und -integrität.
\item Erfüllung des PCI-DSS Standards.
\end{itemize}

\subsection{Hauptkomponenten und Abläufe}

Mit Hilfe der Bizagi-Plattform wurden die folgenden Hauptkomponenten und Abläufe für den SOLL-Prozess modelliert:

\begin{itemize}
\item Integration mit CMDB: Obwohl eine zentrale Datenbank (CMDB) bereits existiert und mit ServiceNow betrieben wird, hat die starke Anpassung von ServiceNow die 
direkte Integration mit Bizagi erschwert. Daher erfolgt die Aktualisierung der CMDB indirekt.
\item **Benachrichtigungssystem:** Bei Abweichungen oder Anmerkungen werden die Mitarbeiter der entsprechenden Fachabteilung informiert. Dies ermöglicht es ihnen, die 
CMDB manuell über ServiceNow zu aktualisieren.
\end{itemize}

\subsection{Modellierung des SOLL-Prozesses mit Bizagi}

Bizagi ist eine renommierte Plattform für die Geschäftsprozessmodellierung und -automatisierung. Die Software bietet verschiedene Werkzeuge zur Darstellung, 
Optimierung und Digitalisierung von Geschäftsprozessen. In diesem Teilabschnitt wird erläutert, wie Bizagi zur Modellierung des SOLL-Prozesses verwendet wurde und welche 
Vorteile die Plattform in diesem Kontext bietet.

\subsubsection{Möglichkeiten von Bizagi}

Bizagi bietet eine Vielzahl von Funktionen, die den gesamten Lebenszyklus eines Geschäftsprozesses abdecken:

\begin{itemize}
\item \textit{Drag-and-Drop-Prozessmodellierung:} Die intuitive Benutzeroberfläche ermöglicht es, Geschäftsprozesse einfach durch Drag-and-Drop zu modellieren, 
wodurch kein technisches Vorwissen erforderlich ist.
\item \textit{Integrierte Datenbank:} Bizagi ermöglicht eine nahtlose Integration mit bestehenden Datenbanken und Systemen.
\item \textit{Formular-Designer:} Mit dem integrierten Formular-Designer können Eingabemasken für Benutzer erstellt werden, um Daten effizient zu erfassen.
\item \textit{Automatisierung und Ausführung:} Einmal modellierte Prozesse können automatisiert und direkt in Bizagi ausgeführt werden.
\item \textit{Berichterstattung und Analyse:} Bizagi bietet leistungsstarke Analysetools, um den Fortschritt und die Leistung von Geschäftsprozessen zu überwachen.
\end{itemize}

\subsubsection{Schritte zur Digitalisierung des Prozesses}

\begin{enumerate}
\item \textit{Analyse des IST-Prozesses:} Vor Beginn der eigentlichen Modellierung wurde der aktuelle, manuelle Prozess analysiert, um Schwachstellen und 
Verbesserungspotenziale zu identifizieren.
\item \textit{Prozessmodellierung:} Mit Hilfe des Bizagi Modelers wurde der SOLL-Prozess Schritt für Schritt modelliert.
\item \textit{Formulargestaltung:} Für den modellierten Prozess wurden entsprechende Eingabemasken mit dem Formular-Designer erstellt.
\item \textit{Integration und Test:} Die Integration des modellierten Prozesses in die bestehende Systemlandschaft wurde durchgeführt, gefolgt von ausführlichen 
Tests zur Sicherstellung der Funktionalität.
\item \textit{Implementierung und Rollout:} Nach erfolgreichen Tests wurde der digitalisierte Prozess in das Live-System überführt und den Nutzern zur Verfügung gestellt.
\end{enumerate}

Durch die Nutzung von Bizagi konnte der neue Geschäftsprozess effizient modelliert, optimiert und schließlich digitalisiert werden. Dabei hat sich besonders der Formular-Designer als wertvolles Werkzeug herausgestellt, da er eine einfache und intuitive Erfassung der benötigten Daten ermöglicht.


\section{Einführung und Abstimmung des neuen Geschäftsprozesses}

Mit einem klar definierten SOLL-Prozess ging es in die Phase der Implementierung und Abstimmung. Dies ist oft die herausforderndste Phase, da sie nicht nur technische, 
sondern auch organisatorische Anpassungen erfordert.

\subsection{Schulungen und Workshops}

Bevor der neue Prozess eingeführt wurde, wurden umfangreiche Schulungen und Workshops für alle beteiligten Mitarbeiter durchgeführt. Ziel war es, sicherzustellen, 
dass jeder die Veränderungen versteht und in der Lage ist, die neuen Tools und Verfahren effektiv zu nutzen, einschließlich des indirekten Aktualisierungsverfahrens für die 
CMDB über ServiceNow.

\subsection{Feedback-Schleifen und Anpassungen}

Nach der ersten Einführung des neuen Prozesses wurden regelmäßige Feedback-Sessions mit den Mitarbeitern durchgeführt. Dies ermöglichte es, eventuelle Schwachstellen 
frühzeitig zu erkennen und den Prozess kontinuierlich zu optimieren.


\end{document}