\documentclass[12pt, a4paper]{article}
\usepackage[german]{babel}

\title{Exposé - Einführung und Digitalisierung eines neuen Geschäftsprozesses}
\author{Kilian Schlosser}
\date{Mai 2023}

\begin{document}
\maketitle
\section{Einleitung}

Im Rahmen dieser Fallstudie wird die Einführung eines neuen Geschäftsprozesses zur Inventarisierung und Sichtprüfung von EFT-Terminals (\textbf{E}lectronic \textbf{F}unds \textbf{T}ransfer) untersucht. 
Derzeit besteht die Herausforderung darin, den bestehenden IST-Prozess aufzunehmen und zu dokumentieren, und einen effizienten SOLL-Prozess zu entwickeln, der eine digitale Lösung zur 
Gewährleistung der Prozessabläufe bietet. Dieses Exposé gibt einen Überblick über die geplanten Schritte zur Digitalisierung des Prozesses unter Verwendung der Bizagi-Plattform,
wobei auch die Frage der Endgeräte (Mobile Datenerfassungsgeräte oder Personal Computer) berücksichtigt wird.

\section{Problemdarstellung}

Der IST-Prozess zur Inventarisierung und Sichtprüfung der EFTs ist derzeit manuell und noch teilweise papierbasiert. Dies führt zu zeitaufwändigen Abläufen, Fehleranfälligkeit und Schwierigkeiten bei der 
Nachverfolgung von EFTs. Um diese Herausforderungen zu bewältigen, soll ein SOLL-Prozess entwickelt werden, der die Digitalisierung des Workflows ermöglicht. Hierbei wird die Bizagi-Plattform als zentrales 
Werkzeug zur Automatisierung und Überwachung der Prozessabläufe eingesetzt.

\section{Digitalisierung mit Bizagi}

Die Digitalisierung des Prozesses erfolgt mithilfe der Bizagi-Plattform, die eine breite Palette von Werkzeugen zur Modellierung, Automatisierung und Analyse von Geschäftsprozessen bietet. 
Wesentliche Ideen hinter der Nutzung von Bizagi sind die Standardisierung der Abläufe, die Reduzierung manueller Eingriffe, die Verbesserung der Datengenauigkeit und die Beschleunigung der Durchlaufzeiten. 
Durch die Verwendung von Bizagi können die EFTs elektronisch in einer Liste erfasst, diese dann automatisch weitergeleitet und überwacht werden.
Weiterhin bietet Bizagi neben der Geschäftsprozessmodellierung auch die Möglichkeit eine Datenbank, die Benutzerschnittstelle sowie Business-Rules zu modellieren.

\section{Modellierung des Geschäftsprozesses}

Die Modellierung des Geschäftsprozesses erfolgt in Bizagi unter Berücksichtigung der spezifischen Anforderungen und Abläufe des Inventarisierungs- und Sichtprüfungsprozesses der EFTs. Hierbei werden die 
verschiedenen Aufgaben, Aktivitäten, Entscheidungspunkte und Verantwortlichkeiten identifiziert und in einem formalen Modell dargestellt. Das Modell dient als Grundlage für die Automatisierung und Überwachung 
des Prozesses.

\section{Einführung und Abstimmung des Prozesses}

Die Einführung des neuen Geschäftsprozesses erfolgt in enger Zusammenarbeit mit den beteiligten Stakeholdern. Es ist wichtig, die relevanten Parteien einzubeziehen, ihre Bedürfnisse zu berücksichtigen und 
mögliche Bedenken oder Widerstände frühzeitig anzugehen. Durch regelmäßige Abstimmungstreffen wird sichergestellt, dass alle Beteiligten den neuen Prozess verstehen und akzeptieren.

\section{Endgerätefrage}

Die Wahl der Endgeräte, die für die Durchführung der Inventarisierung und Sichtprüfung der EFTs verwendet werden, ist ein wichtiger Aspekt bei der Einführung des neuen Geschäftsprozesses. Es gibt verschiedene 
Optionen, darunter Mobile Datenerfassungsgeräte (MDE) und Personal Computer (PC). Die Entscheidung hängt von Faktoren wie der Mobilität der Benutzer, den Anforderungen an die Datenerfassung und der Verfügbarkeit 
von Infrastruktur ab. Eine gründliche Bewertung der Vor- und Nachteile beider Optionen ist erforderlich, um die geeignete Wahl zu treffen.

\end{document}